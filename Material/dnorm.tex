\documentclass{beamer}
\usepackage[spanish]{babel}

\begin{document}


\begin{frame}
    \frametitle{plot} 

Normal {stats}	\hspace{5cm} R Documentation

\begin{center}
The Normal Distribution
\end{center}

{\footnotesize

\textbf{Description}

\vspace{2mm}

Description

Density, distribution function, quantile function and random generation for the normal distribution with mean equal to mean and standard deviation equal to sd.

Usage

dnorm(x, mean = 0, sd = 1, log = FALSE)

pnorm(q, mean = 0, sd = 1, lower.tail = TRUE, log.p = FALSE)

qnorm(p, mean = 0, sd = 1, lower.tail = TRUE, log.p = FALSE)

rnorm(n, mean = 0, sd = 1)
}

\end{frame}

\begin{frame}
    \frametitle{plot} 

Normal {stats}	\hspace{5cm} R Documentation

\begin{center}
The Normal Distribution
\end{center}

{\footnotesize

Arguments

x, q	
vector of quantiles.

p	
vector of probabilities.

n	
number of observations. If length(n) > 1, the length is taken to be the number required.

mean	
vector of means.

sd	
vector of standard deviations.

log, log.p	
logical; if TRUE, probabilities p are given as log(p).

lower.tail	
logical; if TRUE (default), probabilities are P[X � x] otherwise, P[X > x].

}
\end{frame}

\begin{frame}
    \frametitle{plot} 

Normal {stats}	\hspace{5cm} R Documentation

\begin{center}
The Normal Distribution
\end{center}

{\footnotesize

Details

If mean or sd are not specified they assume the default values of 0 and 1, respectively.

The normal distribution has density

\[I
f(x) = 1/(�(2 �) ?) e^-((x - ?)^2/(2 ?^2))
\]

where ? is the mean of the distribution and ? the standard deviation.
}
\end{frame}

\begin{frame}
    \frametitle{The Normal Distribution} 
    
{\footnotesize

Value

dnorm gives the density, pnorm gives the distribution function, qnorm gives the quantile function, and rnorm generates random deviates.

The length of the result is determined by n for rnorm, and is the maximum of the lengths of the numerical arguments for the other functions.

The numerical arguments other than n are recycled to the length of the result. Only the first elements of the logical arguments are used.

For sd = 0 this gives the limit as sd decreases to 0, a point mass at mu. sd < 0 is an error and returns NaN.

Source

For pnorm, based on

Cody, W. D. (1993) Algorithm 715: SPECFUN - A portable FORTRAN package of special function routines and test drivers. ACM Transactions on Mathematical Software 19, 22-32.

For qnorm, the code is a C translation of

Wichura, M. J. (1988) Algorithm AS 241: The percentage points of the normal distribution. Applied Statistics, 37, 477�484.

which provides precise results up to about 16 digits.

For rnorm, see RNG for how to select the algorithm and for references to the supplied methods.

References

Becker, R. A., Chambers, J. M. and Wilks, A. R. (1988) The New S Language. Wadsworth \& Brooks/Cole.

Johnson, N. L., Kotz, S. and Balakrishnan, N. (1995) Continuous Univariate Distributions, volume 1, chapter 13. Wiley, New York.

See Also

Distributions for other standard distributions, including dlnorm for the Lognormal distribution.

}
\end{frame}

\begin{frame}
    \frametitle{The Normal Distribution} 
    
{\footnotesize

\textbf{Examples}

\vspace{2mm}

require(graphics)

\vspace{2mm}

dnorm(0) == 1/sqrt(2*pi)

dnorm(1) == exp(-1/2)/sqrt(2*pi)

dnorm(1) == 1/sqrt(2*pi*exp(1))

\vspace{2mm}

\#\# Using "log = TRUE" for an extended range :

par(mfrow = c(2,1))

plot(function(x) dnorm(x, log = TRUE), -60, 50,
     main = "log { Normal density }")
     
curve(log(dnorm(x)), add = TRUE, col = "red", lwd = 2)

mtext("dnorm(x, log=TRUE)", adj = 0)

mtext("log(dnorm(x))", col = "red", adj = 1)

}
\end{frame}

\begin{frame}
    \frametitle{The Normal Distribution} 
    
{\footnotesize


plot(function(x) pnorm(x, log.p = TRUE), -50, 10,
     main = "log { Normal Cumulative }")
     
curve(log(pnorm(x)), add = TRUE, col = "red", lwd = 2)

mtext("pnorm(x, log=TRUE)", adj = 0)

mtext("log(pnorm(x))", col = "red", adj = 1)

par(mfrow = c(2,1))

\#\# if you want the so-called 'error function'

erf <- function(x) 2 * pnorm(x * sqrt(2)) - 1

\#\# (see Abramowitz and Stegun 29.2.29)

\#\# and the so-called 'complementary error function'

erfc <- function(x) 2 * pnorm(x * sqrt(2), lower = FALSE)

\#\# and the inverses

erfinv <- function (x) qnorm((1 + x)/2)/sqrt(2)

erfcinv <- function (x) qnorm(x/2, lower = FALSE)/sqrt(2)


}
\end{frame}


\begin{frame}
    \frametitle{The Normal Distribution} 
    
{\footnotesize

Examples

require(graphics)

%dnorm(0) == 1/sqrt(2*pi)
%dnorm(1) == exp(-1/2)/sqrt(2*pi)
%dnorm(1) == 1/sqrt(2*pi*exp(1))

\#\# Using "log = TRUE" for an extended range :

par(mfrow = c(2,1))

plot(function(x) dnorm(x, log = TRUE), -60, 50,
     main = "log { Normal density }")
     
curve(log(dnorm(x)), add = TRUE, col = "red", lwd = 2)

mtext("dnorm(x, log=TRUE)", adj = 0)

mtext("log(dnorm(x))", col = "red", adj = 1)

plot(function(x) pnorm(x, log.p = TRUE), -50, 10,
     main = "log { Normal Cumulative }")
     
curve(log(pnorm(x)), add = TRUE, col = "red", lwd = 2)

mtext("pnorm(x, log=TRUE)", adj = 0)

mtext("log(pnorm(x))", col = "red", adj = 1)

\#\# if you want the so-called 'error function'

erf <- function(x) 2 * pnorm(x * sqrt(2)) - 1

\#\# (see Abramowitz and Stegun 29.2.29)

\#\# and the so-called 'complementary error function'

erfc <- function(x) 2 * pnorm(x * sqrt(2), lower = FALSE)

\#\# and the inverses

erfinv <- function (x) qnorm((1 + x)/2)/sqrt(2)

erfcinv <- function (x) qnorm(x/2, lower = FALSE)/sqrt(2)


}
\end{frame}



\end{document}