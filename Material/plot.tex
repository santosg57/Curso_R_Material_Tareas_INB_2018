\documentclass{beamer}
\usepackage[spanish]{babel}

\begin{document}


\begin{frame}
    \frametitle{plot} 

plot {graphics}	\hspace{5cm} R Documentation

\begin{center}
Generic X-Y Plotting
\end{center}

{\footnotesize

\textbf{Description}

\vspace{2mm}

Generic function for plotting of R objects. For more details about the graphical parameter arguments, see {\color{blue} par}.

For simple scatter plots, {\color{blue} plot.default} will be used. However, there are plot methods for many R objects, including {\color{blue} function}S, {\color{blue}data.frame}S, {\color{blue} density} objects, etc. Use methods(plot) and the documentation for these.

\vspace{2mm} 

\textbf{Usage}

plot(x, y, ...)

}

\end{frame}


\begin{frame}
    \frametitle{plot} 

{\footnotesize

\textbf{Arguments}

\textbf{x} - 
the coordinates of points in the plot. Alternatively, a single plotting structure, function or any R object with a plot method can be provided.

\textbf{y} - 
the y coordinates of points in the plot, optional if x is an appropriate structure.

\textbf{...}-
Arguments to be passed to methods, such as graphical parameters (see par). Many methods will accept the following arguments:

}



\end{frame}


\begin{frame}
    \frametitle{plot} 
 
 {\footnotesize
   
\textbf{type} -
what type of plot should be drawn. Possible types are

\begin{itemize}
\item  "p" for points,
\item "l" for lines,
\item "b" for both,
\item "c" for the lines part alone of "b",
\item "o" for both �overplotted�,
\item "h" for �histogram� like (or �high-density�) vertical lines,
\item "s" for stair steps,
\item "S" for other steps, see �Details� below,
\item "n" for no plotting.
\end{itemize}
 }
 
 
 \end{frame} 


\begin{frame}
    \frametitle{plot} 
 
 {\footnotesize
 
\textbf{main} -
an overall title for the plot: see title.

\textbf{sub} -
a sub title for the plot: see title.

\textbf{xlab} -
a title for the x axis: see title.

\textbf{ylab} -
a title for the y axis: see title.

 }
 
 
 \end{frame} 
 
 \begin{frame}
    \frametitle{plot} 
 
 {\footnotesize

\textbf{Details}

The two step types differ in their x-y preference: Going from (x1,y1) to (x2,y2) with x1 $<$ x2, type = "s" moves first horizontal, then vertical, whereas type = "S" moves the other way around.

\vspace{2mm}

\textbf{See Also}

{\color{blue} plot.default, plot.formula} and other methods; {\color{blue}points, lines, par}. For thousands of points, consider using {\color{blue}smoothScatter()} instead of plot().

For X-Y-Z plotting see {\color{blue}contour, persp} and {image}.

 }
 
 \end{frame} 


 \begin{frame}
    \frametitle{plot} 
 
 {\footnotesize

\textbf{Examples}

\vspace{2mm}

require(stats) \# for lowess, rpois, rnorm

plot(cars)

lines(lowess(cars))

\vspace{2mm}

plot(sin, -pi, 2*pi) \# see ?plot.function

\vspace{2mm}

\#\# Discrete Distribution Plot:

plot(table(rpois(100, 5)), type = "h", col = "red", lwd = 10,

     main = "rpois(100, lambda = 5)")

\vspace{2mm}

\#\# Simple quantiles/ECDF, see ecdf() {library(stats)} for a better one:

plot(x <- sort(rnorm(47)), type = "s", main = "plot(x, type = \"s\")")

points(x, cex = .5, col = "dark red")
}
 
 \end{frame} 

\end{document}