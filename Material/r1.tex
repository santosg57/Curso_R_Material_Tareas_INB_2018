\documentclass{article}
% https://es.wikipedia.org/wiki/Regresi%C3%B3n_lineal

\usepackage{amsfonts}
\usepackage{lmodern}
\usepackage[T1]{fontenc}
\usepackage[spanish,activeacute]{babel}
\usepackage{mathtools}

\title{Regresi\'on lineal}
\author{Escribe aqu\'i tu nombre}

\begin{document}
% cuerpo del documento

\maketitle

\begin{center}
   \begin{tabular}{| l | l | }
     \hline
abs & Miscellaneous Mathematical Functions \\ \hline
acos &	Trigonometric Functions \\ \hline
all &	Are All Values True? \\ \hline
any &	Are Some Values True? \\ \hline
apply &	Apply Functions Over Array Margins \\ \hline
acos &	Trigonometric Functions \\ \hline
all &	Are All Values True? \\ \hline
any &	Are Some Values True? \\ \hline
apply &	Apply Functions Over Array Margins \\ \hline
as.data.frame &	Coerce to a Data Frame \\ \hline
as.double &	Double-Precision Vectors \\ \hline
as.factor &	Factors \\ \hline
as.integer	& Integer Vectors \\ \hline
as.list &	Lists - Generic and Dotted Pairs \\ \hline
as.matrix & Matrices \\ \hline
as.numeric &	Numeric Vectors \\ \hline
as.ordered &	Factors \\ \hline
as.table &	Cross Tabulation and Table Creation \\ \hline
as.vector	& Vectors \\ \hline
   \end{tabular}
 \end{center}
 

\begin{center}
   \begin{tabular}{| l | c | r | }
     \hline
     1 & 2 & 3 \\ \hline
     4 & 5 & 6 \\ \hline
     7 & 8 & 9 \\
     \hline
   \end{tabular}
 \end{center}


\end{document}

