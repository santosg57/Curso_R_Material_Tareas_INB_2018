\documentclass{beamer}
\usepackage[spanish]{babel}

\begin{document}

\begin{frame}
    \frametitle{The R Base Package} 

{\scriptsize
\begin{center}
\begin{tabular}{|l|l|}
 \hline
abs & Mathematical Functions \\ \hline
cos &	Trigonometric Functions \\ \hline
all &	Are All Values True? \\ \hline
any &	Are Some Values True? \\ \hline
apply &	Apply Functions Over Array Margins \\ \hline
acos &	Trigonometric Functions \\ \hline
all &	Are All Values True? \\ \hline
any &	Are Some Values True? \\ \hline
apply &	Apply Functions Over Array Margins \\ \hline
as.data.frame &	Coerce to a Data Frame \\ \hline
as.double &	Double-Precision Vectors \\ \hline
as.factor &	Factors \\ \hline
as.integer	& Integer Vectors \\ \hline
as.list &	Lists - Generic and Dotted Pairs \\ \hline
as.matrix & Matrices \\ \hline
as.numeric &	Numeric Vectors \\ \hline
as.ordered &	Factors \\ \hline
as.table &	Cross Tabulation and Table Creation \\ \hline
as.vector	& Vectors \\ \hline
   \end{tabular}
 \end{center}
}

\end{frame}

\begin{frame}
    \frametitle{The R Base Package} 

{\scriptsize
\begin{center}
\begin{tabular}{|l|l|}
 \hline
c	& Combine Values into a Vector or List \\ \hline
cat	& Concatenate and Print \\ \hline
cbind	& Combine R Objects by Rows or Columns \\ \hline
ceiling	& Rounding of Numbers \\ \hline
character	& Character Vectors \\ \hline
choose	& Special Functions of Mathematics \\ \hline
colMeans	& Form Row and Column Sums and Means \\ \hline
colnames	& Row and Column Names \\ \hline
colSums	& Form Row and Column Sums and Means \\ \hline
cos	& Trigonometric Functions \\ \hline
cummax	& Cumulative Sums, Products, and Extremes \\ \hline
cummin	& Cumulative Sums, Products, and Extremes \\ \hline
cumprod	& Cumulative Sums, Products, and Extremes \\ \hline
cumsum	& Cumulative Sums, Products, and Extremes \\ \hline
   \end{tabular}
 \end{center}
}

\end{frame}


\begin{frame}
    \frametitle{The R Base Package} 

{\scriptsize
\begin{center}
\begin{tabular}{|l|l|}
 \hline
data.frame &	Data Frames \\ \hline
data.matrix &	Convert a Data Frame to a Numeric Matrix \\ \hline
det &	Calculate the Determinant of a Matrix \\ \hline
diag &	Matrix Diagonals \\ \hline
dim &	Dimensions of an Object \\ \hline
dir &	List the Files in a Directory/Folder \\ \hline
double &	Double-Precision Vectors \\ \hline
eigen &	Spectral Decomposition of a Matrix \\ \hline
else &	Control Flow \\ \hline
exp &	Logarithms and Exponentials \\ \hline
F &	Logical Vectors \\ \hline
factor &	Factors \\ \hline
factorial &	Special Functions of Mathematics \\ \hline
FALSE &	Logical Vectors \\ \hline
for &	Control Flow \\ \hline
function &	Function Definition \\ \hline
\end{tabular}
\end{center}
}

\end{frame}

\begin{frame}
    \frametitle{The R Base Package} 

{\scriptsize
\begin{center}
\begin{tabular}{|l|l|}
 \hline
getwd  &		Get or Set Working Directory \\ \hline
if  &		Control Flow \\ \hline
ifelse	  &	 Conditional Element Selection \\ \hline
Inf  &		Finite, Infinite and NaN Numbers \\ \hline
integer  &		Integer Vectors \\ \hline
is.array  &		Multi-way Arrays \\ \hline
is.character  &		Character Vectors \\ \hline
is.data.frame  &		Coerce to a Data Frame \\ \hline
is.double  &		Double-Precision Vectors \\ \hline
is.factor  &		Factors \\ \hline
is.finite  &		Finite, Infinite and NaN Numbers \\ \hline
is.infinite  &		Finite, Infinite and NaN Numbers \\ \hline
is.integer  &		Integer Vectors \\ \hline
is.matrix  &		Matrices \\ \hline
is.nan  &		Finite, Infinite and NaN Numbers \\ \hline
is.vector  &		Vectors \\ \hline
\end{tabular}
\end{center}
}

\end{frame}


\begin{frame}
    \frametitle{The R Base Package} 

{\scriptsize
\begin{center}
\begin{tabular}{|l|l|}
 \hline
lapply  &	Apply a Function over a List or Vector \\ \hline
length  &	Length of an Object \\ \hline
letters  &	Built-in Constants \\ \hline
library &	Loading/Attaching and Listing of Packages \\ \hline
log & 	Logarithms and Exponentials \\ \hline
log10 &	Logarithms and Exponentials \\ \hline
mapply &	Apply a Function to Multiple List or Vector Arguments \\ \hline
matrix &	Matrices \\ \hline
max	 & Maxima and Minima \\ \hline
mean  &	Arithmetic Mean \\ \hline
min & Maxima and Minima \\ \hline
\end{tabular}
\end{center}
}

\end{frame}

\begin{frame}
    \frametitle{The R Base Package} 

{\scriptsize
\begin{center}
\begin{tabular}{|l|l|}
 \hline
NA  &		'Not Available' / Missing Values \\ \hline
NaN	 &	Finite, Infinite and NaN Numbers \\ \hline
nchar &		Count the Number of Characters (or Bytes or Width) \\ \hline
ncol	 &	The Number of Rows/Columns of an Array \\ \hline
nrow	 &	The Number of Rows/Columns of an Array \\ \hline
pi	 &	Built-in Constants \\ \hline
q  &	Terminate an R Session \\ \hline
quit	 & Terminate an R Session \\ \hline
range  &	Range of Values \\ \hline
rapply  &	Recursively Apply a Function to a List \\ \hline
rbind	 & Combine R Objects by Rows or Columns \\ \hline
rep  &	Replicate Elements of Vectors and Lists \\ \hline
return  &	Function Definition \\ \hline
\end{tabular}
\end{center}
}

\end{frame}


\begin{frame}
    \frametitle{The R Base Package} 

{\scriptsize
\begin{center}
\begin{tabular}{|l|l|}
 \hline
rev &	Reverse Elements \\ \hline
rm &	Remove Objects from a Specified Environment \\ \hline
round &	Rounding of Numbers \\ \hline
rowsum &	Give Column Sums of a Matrix or Data Frame, Based on a Grouping Variable \\ \hline
rowSums &	Form Row and Column Sums and Means \\ \hline
sample &	Random Samples and Permutations \\ \hline
sapply &	Apply a Function over a List or Vector \\ \hline
seq &	Sequence Generation \\ \hline
setwd &	Get or Set Working Directory \\ \hline
sin &	Trigonometric Functions \\ \hline
sort &	Sorting or Ordering Vectors \\ \hline
source &	Read R Code from a File or a Connection \\ \hline
sqrt &	Miscellaneous Mathematical Functions \\ \hline
sum &	Sum of Vector Elements \\ \hline
svd &	Singular Value Decomposition of a Matrix\\ \hline
\end{tabular}
\end{center}
}

\end{frame}

\begin{frame}
    \frametitle{The R Base Package} 

{\scriptsize
\begin{center}
\begin{tabular}{|l|l|}
 \hline
T  &	Logical Vectors  \\ \hline
t  &	Matrix Transpose  \\ \hline
tapply  &	Apply a Function Over a Ragged Array  \\ \hline
TRUE  &	Logical Vectors  \\ \hline
type  &	The Type of an Object  \\ \hline
typeof  &	The Type of an Object  \\ \hline
unique  &	Extract Unique Elements  \\ \hline
which  &	Which indices are TRUE?  \\ \hline
which.max  &	Where is the Min() or Max() or first TRUE or FALSE ?  \\ \hline
which.min  & Where is the Min() or Max() or first TRUE or FALSE ?  \\ \hline
while  &	Control Flow  \\ \hline
write & Write Data to a File  \\ \hline
\end{tabular}
\end{center}
}

\end{frame}


\end{document}